\setcounter{page}{1}        % 将页码计数器设置为 1

% ==================================================
%
%
%
% --------------------------------------------------

\section{问题重述}

% ==================================================
%
%
%
% --------------------------------------------------

\section{问题分析}

% ==================================================
%
%
%
% --------------------------------------------------

\section{模型假设与符号说明}

\begin{table}[h]
    \centering
    \begin{tabular}{@{}cc@{}}
    \toprule
    符号         & 意义                              \\ \midrule
    $M_f$       & OD客流矩阵                          \\
    $M_f(i,j)$  & 从车站$i$到车站$j$的乘客数量(人)            \\
    $t_i$       & 列车从车站$i$行驶到车站$i+1$所需要的时间($s$) \\
    $t_{avg}$   & 乘客平均上下车时间($s$/人)                  \\
    $P_{in}(i)$         &  在车站$i$上车的人数(人)    \\
    $P_{out}(i)$        & 在车站$i$下车的人数(人)   \\
    $w_i$       & 在车站$i$上下车的总耗时($s$) \\
    $t_{sum}$   & 所有乘客总等待时间($s$)     \\
                &                                 \\
    $M_{in}$    & 乘客上车矩阵           \\
    $M_{in}(i,j)$    & 列车在站台$i$时乘客在站台$j$下车的实际上车人数(人)     \\
    $V_{out}$    & 乘客下车向量           \\
    $M_{out}(i)$    & 乘客在站台$i$的实际下车人数(人)     \\
    $M_{cur}$    & 列车内乘客矩阵           \\
    $M_{cur}(i,j)$    & 列车刚到站台$i$时, 其内乘客到站台$j$的人数(人)     \\
                &                                 \\
                &                                 \\
                &                                 \\ \bottomrule
    \end{tabular}
\end{table}
